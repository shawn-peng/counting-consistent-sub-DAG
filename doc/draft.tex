\documentclass[11pt]{article}
\usepackage{amsmath,amsthm,amssymb}
\usepackage[top=1in,bottom=1in,left=1.25in,right=1.25in]{geometry}
\usepackage[linesnumbered,lined,boxed,commentsnumbered,longend]{algorithm2e}

\theoremstyle{definition}
\newtheorem{definition}{Definition}[section]

\newcommand{\subg}[2]{\ensuremath{{#1}[#2]}}
\newcommand{\colg}[2]{\ensuremath{{#1}^{\left(#2\right)}}}
\newcommand{\ance}[1]{\ensuremath{\mathcal{A}_{#1}}}
\newcommand{\desc}[1]{\ensuremath{\mathcal{D}_{#1}}}

\title{On the consistent sub-DAG counting problem}
\author{Yuxiang Jiang}
\date{\today}
\begin{document}
\maketitle
\section{Notation}
    \begin{table}[htp!]
        \renewcommand\arraystretch{1.5}
        \begin{tabular}{c|l}
            \hline\hline
            Notation & Meaning \\ \hline
            $G = (V, E)$ & A directed graph $G$ made up of a vertex set $V$ and an edge set $E$.\\
            $\mathcal{G}$ & A set of directed graphs.\\
            $u$, $v$, \textit{etc.} & Vertices in a graph.\\
            $(u, v)$ & An ordered pair which denotes a directed edge $u \to v$ in the set $E$.\\
            $\subg{G}{S}$ & A vertex-induced sub-graph of $G$, where $S \subseteq V$.\\
            $\subg{G}{-S}$ & A sub-graph of $G$ induced by vertices that are not in $S$.\\
            $\colg{G}{S}$ & A derived graph from $G$ by collapsing vertices in $S$ into a virtual vertex.\\
            $\tilde{v}$ & A collapsed virtual vertex.\\
            $\ance{S}$ & The union of $S$ and all ancestors of $u \in S$ in $G$.\\
            $\desc{S}$ & The union of $S$ and all descendants of $u \in S$ in $G$.\\
            $\ance{u_i}, \desc{u_i}$ & Shorthand notations for $\ance{\{u_i\}}$
            and $\desc{\{u_i\}}$.\\
            \hline
        \end{tabular}
    \end{table}

\section{The consistent sub-DAG counting problem}
    In this problem, we are given a \emph{directed acyclic graph}~(DAG)
    $G=(V,E)$, which potentially can have more than roots (\textit{i.e.}
    vertices with no incoming edges) and disconnected. We are then interested in
    a particular type of vertex-induced sub-graphs $\subg{G}{S}$ which we call
    \emph{consistent sub-DAG}, and would like to know the number of all such
    sub-DAGs.  \begin{definition} A sub-graph $\subg{G}{S}$ is called
        \emph{consistent} \textit{iff} $\forall v \in S$, $(u, v) \in E \implies
        u \in S$.  \end{definition} To precisely compute the size of the
    collection $\mathcal{G} = \left\{\subg{G}{S} \mid \subg{G}{S} \mbox{ is
            consistent}\right\}$, we define a sub-routine $f$ which takes two
    inputs: (i) a DAG $G$ and (ii) a subset $S$ of its vertices and returns the
    number of consistent sub-DAGs each of which contains $S$. For example,
    \dots. Therefore, the original problem reduces to compute $f(G,
    \varnothing)$.

\section{Algorithm}
    The algorithm works in an induction manner, which can be described in the
    following recursive function. We start from an empty graph $\varnothing$
    (aka.  $G_0$) and keep adding vertices one at a time (with its incoming
    edges) in a topological order. Let $u_1, u_2, \ldots, u_n \in V$ be a sorted
    order, and we use $G_i = (V_i, E_i), i = 1, \ldots, n$ to denote the
    transient graph after adding back $u_i$. And during this reconstruction
    process, we keep track of those counts $f(G_i, \varnothing)$ on-the-fly to
    get $f(G_n, \varnothing)$ in the end as the result. Note that the subscript
    $i$ of $G_i$ coincides with the size of the graph.

    \SetKwFunction{CCSD}{Count-cDAG}
    \begin{algorithm}[htp!]
        \SetKwProg{Fn}{Function}{}{end}
        \Fn(\tcc*[h]{equivalent to compute $f(G, \varnothing)$}){\CCSD{$G$}}{
            \If{$G=\varnothing$}{
                \KwRet 1\;\label{algo:trivial}
            }
            Topologically sort vertices in $V$ to get an order: $u_1, u_2, \ldots,
            u_n$\;
            $f(G_0, \varnothing) \leftarrow 1$\;
            \For{$i \leftarrow 1$ \KwTo $n$}{
                \uIf{$u_i$ has no parent}{
                    $f(G_i, \varnothing) = 2 \times f(G_{i-1},
                    \varnothing)$\;\label{algo:case0}
                }
                \Else{
                    $f(G_i, \varnothing) = \CCSD{$\colg{G_i}{\ance{u_i}}$}
                    - \CCSD{$\subg{G_i}{-\desc{\ance{u_i}}}$}
                    + f(G_{i-1}, \varnothing)
                    $\;\label{algo:case1}
                }
            }
            \KwRet $f(G_n, \varnothing)$\;
        }
        \caption{Count the number of consistent sub-DAGs of a graph.}
    \end{algorithm}

    First, we define $f(\varnothing, \varnothing)$ to be $1$ for the
    \emph{trivial} empty graph (Line~\ref{algo:trivial}) as we consider an empty
    graph is a consistent sub-DAG of any graph. Line~\ref{algo:case0} handles
    the case where $u_i$ is a root vertex, which turns out to be a singleton in
    $G_i$. Therefore, the numer of cDAGs of the current graph $G_i$ is twice as
    many as that of $G_{i-1}$, since $u_i$ is independent of the rest.

    The most involved part of this algorithm is Line~\ref{algo:case1}, which
    handles the cases where $u_i$ has at least one parent vertices. All cDAGs of
    $G_i$ can be divided into two non-overlapping groups: one includes $u_i$ and
    the other excludes $u_i$. The latter is indeed $f(G_{i-1}, \varnothing)$
    which has been already computed; while the former, $f(G_i, \{u_i\})$, is
    equivalent to $f(G_i, \ance{u_i})$, since all vertices in $\ance{u_i}$ have
    to be included given the inclusion of $u_i$ due to consistency. We can
    ``collapse'' $\ance{u_i}$ to be a single ``virtual'' vertex (denoted as
    $\tilde{v}$), and notice that there is no incoming edges for this vertex in
    the collapsed graph. In other words, $\tilde{v}$ is the root (or at least
    one of the roots) of $\colg{G_i}{\ance{u_i}}$. Now, to compute
    $f(\colg{G_i}{\ance{u_i}}, \{\tilde{v}\})$, we can recursively call the
    function \CCSD on graph $\colg{G_i}{\ance{u_i}}$ and
    $\subg{G_i}{-\desc{\ance{u_i}}}$ respectively, and subtract the result of
    one from the other. Recall that \CCSD{$\colg{G_i}{\ance{u_i}}$} gives the
    number of cDAGs including those without selecting $\tilde{v}$.
    These extra counts are given by \CCSD{$\subg{G_i}{-\desc{\ance{u_i}}}$},
    since these sub-DAGs are the only ones that contribute to valid counts in
    $\colg{G_i}{\ance{u_i}}$ without selecting $\tilde{v}$. Note that, in the
    case of single-root DAGs, this term can be simplified as
    \begin{align*}
        \CCSD{$\subg{G_i}{-\desc{\ance{u_i}}}$} = \CCSD{$\varnothing$} = 1.
    \end{align*}

    That is for the number of consistent sub-DAGs in $G_i$ that contains $u_i$, we have
    \begin{align*}
        f(G_i, \{u_i\}) &= f(G_i, \ance{u_i})\\
        &= f(\colg{G_i}{\ance{u_i}}, \varnothing) -
           f(\subg{G_i}{-\desc{\ance{u_i}}}, \varnothing)\\
        &= \CCSD{$\colg{G_i}{\ance{u_i}}$} -
           \CCSD{$\subg{G_i}{-\desc{\ance{u_i}}}$}.
    \end{align*}

    Finally, because both $\left|V_i \bigcap \ance{u_i}\right|$ and $\left|V_i
    \bigcap \desc{\ance{u_i}}\right|$ are at least 2, we have the size of
    these two sub-problems:
    \begin{align*}
        \left|\colg{G_i}{\ance{u_i}}\right| = \left|\subg{G_{i}}{-\ance{u_i}}\right| + 1 &\leq i - 1
        \quad\mbox{and}\\
        \left|\subg{G_i}{-\desc{\ance{u_i}}}\right| &< i-1,
    \end{align*}
    which guarantees that the algorithm stops in a finite number of recursive calls.

\section{Analysis}
    TBA.

\end{document}
